\documentclass[10pt]{article}
\usepackage{hyperref}
\usepackage{amsmath}
\usepackage{amsfonts}
\usepackage{minted}
\usepackage{tikz}
\usetikzlibrary{arrows,backgrounds,shapes,matrix,positioning,fit,calc}
\usetikzlibrary{decorations.pathreplacing,angles,quotes}
\usetikzlibrary{arrows.meta}
\usetikzlibrary{calendar}
\usetikzlibrary{trees}


\title{BitcoinVM}
\date{August 8, 2022}
\author{Saravanan Vijayakumaran}

\begin{document}
\maketitle
\begin{abstract}
  A BitcoinVM is a circuit which maintains the set of all Bitcoin UTXOs in a sparse Merkle tree (SMT). Each new Bitcoin block will be used to update the SMT using a validity proof. The SMT can be used to generate privacy-preserving proofs of ownership of a certain amount of bitcoins. Such proofs will enable privacy-preserving proofs of solvency of cryptocurrency exchanges.
\end{abstract}
  
\section{Motivation}%
\label{sec:motivation}
Cryptocurrency exchanges may engage in fractional-reserve operations, where they sell more bitcoins to their customers than they own. Provisions \cite{Dagher2015} was the first privacy-preserving proof of solvency protocol to address this issue.

A shortcoming of the Provisions protocol was that it could only include bitcoins stored in \textit{Pay To Public Key (P2PK)} and \textit{Pay To Public Key Hash (P2PKH)} addresses, with the additional constraint that the preimage of the P2PKH addresses must be known. Even before SegWit activation, Bitcoin supported two additional address types: \textit{$m$-of-$n$ Multi-signature (multisig)} and \textit{Pay to Script Hash (P2SH)} adddresses \cite[Section 5.5]{Vijayakumaran2017}.

Multisig addresses offer better security and flexibility by requiring signatures from any $m$ out of $n$ public keys to spend a UTXO. Multisig addresses require all $n$ keys to be explicitly specified in a UTXO, thereby increasing the transaction cost of sending bitcoins to a multisig address.\footnote{Bitcoin transaction fees is proportional to the length of the transaction.} P2SH addresses solve this issue by requiring that only a 20 byte hash (SHA256 + RIPEMD160) of a multisig address be specified to receive funds into it.

To the best of our knowledge, there is no publicly verifiable privacy-preserving proof of reserves protocol for Bitcoin that supports all the address types. The Kraken exchange uses a third-party auditor to help run their proof of reserves protocol \cite{KrakenPoR}. In Feb 2020, Kraken CEO Jesse Powell mentioned lack of multisig address support in Provisions as one of the reasons to defer attempting its use \cite{KrakenCEO}.

\textbf{Goal:} Enable privacy-preserving proof of Bitcoin reserves that is publicly verifiable on-chain. 

\section{BitcoinVM}%
\label{sec:bitcoinvm}
To prove ownership of Bitcoin UTXOs in a privacy-preserving manner, they have to be stored in data structure that is amenable to zero-knowledge proofs. Our initial thoughts are as follows:
\begin{enumerate}
  \item Store all the Bitcoin UTXOs upto a certain block height in a sparse Merkle tree (SMT). The key to the leaves could be a hash of the transaction ID (TxID) and output index that identifies a UTXO.
  \item Each Bitcoin block has transactions that spend old UTXOs and create new ones. The SMT would need to be updated after each block.
  \item Transaction validation in Bitcoin involves feeding a challenge script and its corresponding response script as inputs to a stack machine which supports a small number of opcodes. 
  \item The BitcoinVM will be the circuit that gives a validity proof of the correctness of SMT root transitions. It needs to be able to verify scripts written in Bitcoin Script. 
  \item Given a valid SMT with UTXOs as its leaves, ownership of a certain amount of coins can be proven using privacy-preserving Merkle proofs (like in Tornado Cash).
\end{enumerate}

\section{Challenges}%
\label{sec:challenges}
BitcoinVM needs several circuits not currently present in the Halo2 ecosystem.
\begin{itemize}
  \item RIPEMD160 hash function
  \item Bitcoin Script validator
    \begin{itemize}
      \item Multisig response script validator
      \item P2SH response script validator
      \item SegWit response script validator
    \end{itemize}
  \item Sparse Merkle Tree (??)
  \item And probably many more
\end{itemize}



\newpage
\bibliographystyle{unsrt}
\bibliography{bvm}
\end{document}
