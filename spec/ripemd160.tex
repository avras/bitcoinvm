\documentclass[10pt]{article}
\usepackage{hyperref}
\usepackage{amsmath}
\usepackage{amsfonts}
\usepackage{minted}
\usepackage{tikz}
\usepackage[a4paper]{geometry}
\usetikzlibrary{arrows,backgrounds,shapes,matrix,positioning,fit,calc}
\usetikzlibrary{decorations.pathreplacing,angles,quotes}
\usetikzlibrary{arrows.meta}
\usetikzlibrary{calendar}
\usetikzlibrary{trees}


\title{RIPEMD160 Gadget Implementation using the \texttt{halo2} API}
\author{Saravanan Vijayakumaran}

\begin{document}
\maketitle
  
\section{Motivation}%
\label{sec:motivation}
The HASH160 opcode in the Bitcoin protocol hashes a given byte sequence with the SHA256 hash function followed by the RIPEMD160 hash function. It is used to obtain 20-byte hashes of objects like public keys and Bitcoin scripts.

To prove the validity of Bitcoin response scripts for Pay-to-Public-Key (P2PK) and Pay-to-Script-Hash (P2SH) UTXOs using the \texttt{halo2} API, we need an implementation of a HASH160 gadget.
A SHA256 gadget implementation already exists in the \href{https://github.com/zcash/halo2/tree/main/halo2_gadgets/src}{\texttt{zcash/halo2}} repository. In this document, we describe the design and implementation of a RIPEMD160 gadget. It is heavily inspired by SHA256 gadget implementation.

\section{RIPEMD160 Specification}%
\label{sec:ripemd160_specification}
A specification of the RIPEMD160 hash function is available at \url{https://homes.esat.kuleuven.be/~bosselae/ripemd160.html}, along with a C implementation. We repeat it here for convenience and for fixing the notation use to describe the gadget implementation. 



\newpage
\bibliographystyle{unsrt}
\bibliography{bvm}
\end{document}
